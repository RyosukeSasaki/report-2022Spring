\documentclass[uplatex,a4j,11pt,dvipdfmx]{jsarticle}
\usepackage{listings,jvlisting}
\bibliographystyle{junsrt}

\usepackage{url}

\usepackage{graphicx}
\usepackage{gnuplot-lua-tikz}
\usepackage{pgfplots}
\usepackage{tikz}
\usepackage{amsmath,amsfonts,amssymb}
\usepackage{bm}
\usepackage{siunitx}

\makeatletter
\def\fgcaption{\def\@captype{figure}\caption}
\makeatother
\newcommand{\setsections}[3]{
\setcounter{section}{#1}
\setcounter{subsection}{#2}
\setcounter{subsubsection}{#3}
}
\newcommand{\mfig}[3][width=15cm]{
\begin{center}
\includegraphics[#1]{#2}
\fgcaption{#3 \label{fig:#2}}
\end{center}
}
\newcommand{\gnu}[2]{
\begin{figure}[hptb]
\begin{center}
\input{#2}
\caption{#1}
\label{fig:#2}
\end{center}
\end{figure}
}

\makeatletter
\def\WordCount#1{%
\@tempcnta\z@%
\@tfor \@tempa:=#1\do{\advance\@tempcnta\@ne}%
(\the\@tempcnta 文字)\\
{#1}%
}
\makeatother

\begin{document}
\title{}
\author{61908697 佐々木良輔}
\date{}
\maketitle
\noindent2246 文字 (見出しを含まない)
\subsection*{序論}
現代社会は科学技術と深く結びついている.
特に科学技術を社会実装しようとすると,利益だけではなく様々な弊害や利害関係の衝突を引き起こす.
こういった問題に対しては科学技術だけではなく政治による意思決定がなされる必要があり,
このような科学と政治が関わる諸問題をトランス・サイエンスと呼ぶ.\cite{class}

トランス・サイエンスの問題がありふれた現代において科学者は,
科学で得られる知識を政治に対して提示し,必要であれば政治での議論に参加しながら問題を解決していく必要がある.
同時に,トランス・サイエンスの問題に関わる場合であっても
科学者は科学によって生産される知識の質を担保,向上するよう務めるべきであると考える.
\subsection*{本文}
科学技術は現代社会を維持する上で必要不可欠な存在である.
社会は科学に対して社会問題の解決や社会の向上を期待していると考えられ,
科学は今後も発展を続けるためにもこの期待から目を背けるべきではない.
しかし現代の社会問題は科学だけで答えを提示できるケースばかりではなく,
トランス・サイエンス的な問題に対しても科学者は取り組んで行く必要がある.
このようなトランス・サイエンスの時代において
科学の発展のために科学者が研究以外に行うべきこととして以下の3点を挙げる.
\begin{enumerate}
  \item 社会問題の認知
  \item 科学とトランス・サイエンスの境界の提示
  \item トランス・サイエンスの問題に関する議論への参加
\end{enumerate}
\subsubsection*{1. 社会問題の認知}
トランス・サイエンス的な問題は,科学を社会に能動的に組み込もうとする際に発生する可能性もあるが,
一方で既に存在する問題を科学技術で解決できる場合や,
そもそもこれまで問題と認識されていなかったという場合も考えられる.
いずれにせよ,トランス・サイエンス的な問題に取り組むためには科学者自身がどこにどのような問題があるかを認知し,
その中で自らの専門性が活かせそうな問題について議論に参加することが必要になると考える.

これまで問題と認識されていなかったものが科学技術によって改善した例として,近年のデジタル化が挙げられる.
デジタル化によってこれまでの社会に内在していた様々な非効率な作業が効率化されている.
また,近年では企業や組織をするにあたり,技術的な問題やセキュリティなどについて補助を行うコンサルタントなどの業務も台頭している.
これはまさにこれまで問題とされていなかった社会問題を認知するようになったことで様々な業務が改善できた例と言える.
このようにデジタル化が社会の様々な問題を改善していった結果,
IT技術に対する社会の期待は非常に高まり,情報分野が人材や予算の獲得,
ひいては近年の急発展に成功したことは明らかである.

以上から,科学者はまず社会問題を認知し,自らの専門性を発揮できる問題に取り組むことが
存在感の向上や科学自体の発展につながると考える.
\subsubsection*{2. 科学とトランス・サイエンスの境界の提示}
科学者が社会問題を認知し政治や市民との議論を行う場合,
科学者は科学に基づいた知見を提示し,その知見を考慮して政治との議論を行うことになる.
ここで科学者は自らの持つ知識や今後の研究で得られるであろう知識から科学に基づいて判断の行える領域と,
その先にある政治的な判断が必要な領域との境界を可能な限り明確にしておく必要があると考える.
このような科学とトランス・サイエンスの境界線を引く作業をBoundary Workと呼ぶ.\cite{trans:online}
このBoundary Workが必要と考える理由は以下の2つである.

1つ目の理由として,境界が曖昧なまま議論が進んだ場合,
科学の裁量が実際の判断能力を超えて与えられることで市民や政治による議論が不十分なまま判断が行われたり,
あるいは逆に政治や市民による意思決定が科学に基づかないものになってしまう事が考えられる.
また実際には政治によって為された判断が科学によるものだと誤認されることも避けるべきである.

2つ目の理由として科学の裁量が著しく損なわれた場合,科学により生産される知識そのものに政治の意向が反映されてしまうといった事態も想定できる.
科学は積み重ねの学問であり,事実と反する知識が一度事実として受け入れられてしまうと
その後の研究によって生み出された知識や社会実装に対して時限爆弾を仕込むことになる.

以上のような理由から,科学者は政治との適切な議論や意思決定を行うため,
また科学の成果の品質を担保するためにも科学とトランス・サイエンスの境界を明確に提示するべきである.
\subsubsection*{3. トランス・サイエンスの問題に関する議論への参加}
前節で述べたような境界線が策定できたとしても,科学者は知識を提示するだけでなく市民による議論にも参画すべきである.
その理由として以下の2つがある.

1つ目の理由として,まず科学によって提示された知識が誤った理解をされないようにする必要がある.
科学によって提示された知識が政治の議論によって誤って理解され,それに基づいて意思決定が為されることは
科学による知識が議論をミスリードしているということであり,
これを正さないことは専門家として道義的に許されないと考える.

2つ目の理由として,政治や市民の議論において科学の存在感を維持することがある.
科学が継続的に発展するためには,科学の成果が広く認識される必要がある.
そのために科学者は市民として,そして専門家として議論に参加し,科学の成果をアピールするべきであると考える.

以上の理由から,科学者は科学的な知識を提示するだけでなく,議論の場にも積極的に参加するべきであると考える.
\subsection*{結論}
以上の議論から,トランス・サイエンスの時代において科学の必要性をアピールし科学が発展し続けるために必要なこととして,
科学者は社会に内在する問題を明らかにし,その立場を明らかにした上で政治に対して議論の材料を提示し,
更には政治での議論に参画するといったことがあると考えられる.
\bibliography{ref.bib}
\end{document}