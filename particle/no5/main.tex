\documentclass[uplatex,a4j,11pt,dvipdfmx]{jsarticle}
\usepackage{listings,jvlisting}
\bibliographystyle{junsrt}

\usepackage{url}

\usepackage{graphicx}
\usepackage{gnuplot-lua-tikz}
\usepackage{pgfplots}
\usepackage{tikz}
\usepackage{amsmath,amsfonts,amssymb}
\usepackage{bm}
\usepackage{siunitx}

\makeatletter
\def\fgcaption{\def\@captype{figure}\caption}
\makeatother
\newcommand{\setsections}[3]{
\setcounter{section}{#1}
\setcounter{subsection}{#2}
\setcounter{subsubsection}{#3}
}
\newcommand{\mfig}[3][width=15cm]{
\begin{center}
\includegraphics[#1]{#2}
\fgcaption{#3 \label{fig:#2}}
\end{center}
}
\newcommand{\gnu}[2]{
\begin{figure}[hptb]
\begin{center}
\input{#2}
\caption{#1}
\label{fig:#2}
\end{center}
\end{figure}
}

\begin{document}
\title{素粒子物理学 宿題5}
\author{61908697 佐々木良輔}
\date{}
\maketitle
人間の密度を$1\ \si{\gram.\centi\metre^{-3}}$とする.
中性子の質量が$1.674\times10^{-24}\ \si{\gram}$,
陽子の質量が$1.672\times10^{-24}\ \si{\gram}$
であり人間には中性子と陽子が同数が含まれるとすると,
核子の質量は平均して$1.673\times10^{-24}\ \si{\gram}$となる.\cite{n:online}\cite{p:online}
電子の質量を無視すれば人間の$1\ \si{\centi\metre^3}$に含まれる核子の数は
\begin{align*}
  \frac{1}{1.673\times10^{-24}}\simeq5.977\times10^{23}\ \si{\centi\metre^{-3}}
\end{align*}
である.
$10\ \si{\giga\electronvolt}$のニュートリノと核子の散乱の断面積は$\sigma\sim10^{-37}\ \si{\centi\metre^2}$なので
人間中でのニュートリノの平均自由工程$l$は
\begin{align*}
  l=\frac{1}{5.977\times10^{23}\ \si{\centi\metre^{-3}}\times10^{-37}\ \si{\centi\metre^2}}\simeq1.673\times10^{13}\ \si{\centi\metre}
\end{align*}
となる.一方で60億人の人間を一列に並べた際の長さ$L$は
\begin{align*}
  L=50\ \si{\centi\metre}\times6.0\times10^9=3.0\times10^{11}\ \si{\centi\metre}
\end{align*}
なのでニュートリノが散乱する確率は
\begin{align*}
\frac{L}{l}=1.793\times10^{-2}
\end{align*}
となる.
\bibliography{ref.bib}
\end{document}