\documentclass[uplatex,a4j,11pt,dvipdfmx]{jsarticle}
\usepackage{listings,jvlisting}
\bibliographystyle{junsrt}

\usepackage{url}

\usepackage{graphicx}
\usepackage{gnuplot-lua-tikz}
\usepackage{pgfplots}
\usepackage{tikz}
\usepackage{amsmath,amsfonts,amssymb}
\usepackage{bm}
\usepackage{siunitx}

\makeatletter
\def\fgcaption{\def\@captype{figure}\caption}
\makeatother
\newcommand{\setsections}[3]{
\setcounter{section}{#1}
\setcounter{subsection}{#2}
\setcounter{subsubsection}{#3}
}
\newcommand{\mfig}[3][width=15cm]{
\begin{center}
\includegraphics[#1]{#2}
\fgcaption{#3 \label{fig:#2}}
\end{center}
}
\newcommand{\gnu}[2]{
\begin{figure}[hptb]
\begin{center}
\input{#2}
\caption{#1}
\label{fig:#2}
\end{center}
\end{figure}
}

\begin{document}
\title{素粒子物理学 宿題2}
\author{61908697 佐々木良輔}
\date{}
\maketitle
\subsubsection*{水の質量}
寿命が$10^{33}\ {\rm years}$の陽子の崩壊を平均$1\ 回/{\rm year}$の頻度で観測するには
$10^{33}$個の陽子を用意する必要がある.
1個の水分子には2個のFree Protonが含まれるので,必要な水分子の数は$10^{33}/2$個である.
この水の質量は
\begin{align}
  \frac{10^{33}}{2}\times\frac{18.0\ \si{\gram.\mole^{-1}}}{N_A\ \si{\mole^{-1}}}\times\frac{1}{10^6}=1.49\times10^4\ \si{\tonne}
\end{align}
となる.またその体積は
\begin{align}
  1.49\times10^4\ \si{\metre^3}
\end{align}
となる.
\subsubsection*{PMTの本数}
ここでは前問で求めた量の水を球状のタンクに収めるとする.この球状タンクの半径$r$は
\begin{align}
  \frac{4}{3}\pi r^3&=1.49\times 10^4\nonumber\\
  r&=15.3\ \si{\metre}
\end{align}
その表面積は
\begin{align}
  4\pi r^2=2.93\times 10^3\ \si{\metre^2}
\end{align}
また直径$50\ \si{\centi\metre}$のPMTの断面積は
\begin{align}
  \pi\times0.25^2=0.196\ \si{\metre^2}
\end{align}
なので,光被覆率を$10\%$とするには
\begin{align}
  \frac{2.93\times 10^3}{0.196}=1.49\times10^4\ 個
\end{align}
のPMTが必要になる.
\end{document}