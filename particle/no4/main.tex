\documentclass[uplatex,a4j,11pt,dvipdfmx]{jsarticle}
\usepackage{listings,jvlisting}
\bibliographystyle{junsrt}

\usepackage{url}

\usepackage{graphicx}
\usepackage{gnuplot-lua-tikz}
\usepackage{pgfplots}
\usepackage{tikz}
\usepackage{amsmath,amsfonts,amssymb}
\usepackage{bm}
\usepackage{siunitx}

\makeatletter
\def\fgcaption{\def\@captype{figure}\caption}
\makeatother
\newcommand{\setsections}[3]{
\setcounter{section}{#1}
\setcounter{subsection}{#2}
\setcounter{subsubsection}{#3}
}
\newcommand{\mfig}[3][width=15cm]{
\begin{center}
\includegraphics[#1]{#2}
\fgcaption{#3 \label{fig:#2}}
\end{center}
}
\newcommand{\gnu}[2]{
\begin{figure}[hptb]
\begin{center}
\input{#2}
\caption{#1}
\label{fig:#2}
\end{center}
\end{figure}
}

\begin{document}
\title{素粒子物理学 宿題4}
\author{61908697 佐々木良輔}
\date{}
\maketitle
粒子はMWPCに対して一様な分布で入射するものとする.
すなわち位置$x\in[-d/2,d/2]$に粒子が入射する確率は一様に$1/d$とする.
このとき位置の分散$\sigma_x^2$は
\begin{align*}
  \sigma_x^2=\langle x^2\rangle-\langle x\rangle^2&=\int_{-d/2}^{d/2}\frac{1}{d}x^2dx-\left(\int_{-d/2}^{d/2}\frac{1}{d}xdx\right)^2\\
  &=\left[\frac{1}{d}\frac{x^3}{3}\right]_{-d/2}^{d/2}-0\\
  &=\frac{d^2}{12}
\end{align*}
したがって標準偏差$\sigma_x$は
\begin{align*}
  \sigma_x=\frac{d}{\sqrt{12}}
\end{align*}
となる.
\bibliography{ref.bib}
\end{document}