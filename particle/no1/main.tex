\documentclass[uplatex,a4j,11pt,dvipdfmx]{jsarticle}
\usepackage{listings,jvlisting}
\bibliographystyle{junsrt}

\usepackage{url}

\usepackage{graphicx}
\usepackage{gnuplot-lua-tikz}
\usepackage{pgfplots}
\usepackage{tikz}
\usepackage{amsmath,amsfonts,amssymb}
\usepackage{bm}
\usepackage{siunitx}

\makeatletter
\def\fgcaption{\def\@captype{figure}\caption}
\makeatother
\newcommand{\setsections}[3]{
\setcounter{section}{#1}
\setcounter{subsection}{#2}
\setcounter{subsubsection}{#3}
}
\newcommand{\mfig}[3][width=15cm]{
\begin{center}
\includegraphics[#1]{#2}
\fgcaption{#3 \label{fig:#2}}
\end{center}
}
\newcommand{\gnu}[2]{
\begin{figure}[hptb]
\begin{center}
\input{#2}
\caption{#1}
\label{fig:#2}
\end{center}
\end{figure}
}

\begin{document}
\title{素粒子物理学 レポート1}
\author{61908697 佐々木良輔}
\date{}
\maketitle
$E_\mu$, $E_{\nu_\mu}$を$\mu$, $\nu_\mu$の全エネルギーとすれば
\begin{align}
  m_\pi^2=(E_\mu+E_{\nu_\mu})^2-|\vec{p}_\mu+\vec{p}_{\nu_\mu}|
\end{align}
ここで$\pi$粒子は静止していたので$\vec{p}_\mu+\vec{p}_{\nu_\mu}=0$より
\begin{align}
  \begin{split}
    m_\pi^2&=(E_\mu+E_{\nu_\mu})^2\\
    &=\left(\sqrt{m_\mu^2+p_\mu^2}+\sqrt{m_{\nu_\mu}^2+p_{\nu_\mu}^2}\right)^2\
  \end{split}
\end{align}
ここで$|\vec{p}_{\mu}|=|\vec{p}_{\nu_\mu}|$より
\begin{align}
  \begin{split}
    m_\pi^2&=m_\mu^2+2p_\mu^2+2\sqrt{\left(m_\mu^2+p_\mu^2\right)p_\mu^2}\\
    m_\pi^2-m_\mu^2-2p_\mu^2&=2\sqrt{\left(m_\mu^2+p_\mu^2\right)p_\mu^2}\\
    4182\ \si{\mega\electronvolt^2}-p_\mu^2&=\sqrt{(106^2\ \si{\mega\electronvolt^2}+p_\mu^2)p_\mu^2}\\
    \left(4182\ \si{\mega\electronvolt^2}-p_\mu^2\right)^2&=(106^2\ \si{\mega\electronvolt^2}+p_\mu^2)p_\mu^2\\
    0&=4182^2\ \si{\mega\electronvolt^4}-(2\cdot4182+106^2)\ \si{\mega\electronvolt^2}\cdot p_\mu^2\\
    p_\mu&=\pm 29.8\ \si{\mega\electronvolt}
  \end{split}
\end{align}
となる.
\end{document}