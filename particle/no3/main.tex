\documentclass[uplatex,a4j,11pt,dvipdfmx]{jsarticle}
\usepackage{listings,jvlisting}
\bibliographystyle{junsrt}

\usepackage{url}

\usepackage{graphicx}
\usepackage{gnuplot-lua-tikz}
\usepackage{pgfplots}
\usepackage{tikz}
\usepackage{amsmath,amsfonts,amssymb}
\usepackage{bm}
\usepackage{siunitx}

\makeatletter
\def\fgcaption{\def\@captype{figure}\caption}
\makeatother
\newcommand{\setsections}[3]{
\setcounter{section}{#1}
\setcounter{subsection}{#2}
\setcounter{subsubsection}{#3}
}{
\newcommand{\mfig}[3][width=15cm]{
\begin{center}
\includegraphics[#1]{#2}
\fgcaption{#3 \label{fig:#2}}
\end{center}
}
\newcommand{\gnu}[2]{
\begin{figure}[hptb]
\begin{center}
\input{#2}
\caption{#1}
\label{fig:#2}
\end{center}
\end{figure}
}

\begin{document}
\title{素粒子物理学 宿題3}
\author{61908697 佐々木良輔}
\date{}
\maketitle
\subsection*{(1)}
ガスチェレンコフカウンタで$10\ \si{\giga\electronvolt}$の$\pi^\pm$と$K^\pm$を識別するには図\ref{fig:fig/fig1.jpg}の
青線の範囲に$1/n$を収めれば$\pi^\pm$のみがチェレンコフ放射を起こすため識別できる.
ここで$10\ \si{\giga\electronvolt}$における$\pi^\pm$と$K^\pm$の$\beta$はそれぞれ
\begin{align*}
  \beta_{\pi^\pm}=\frac{p}{\sqrt{m^2+p^2}}=\frac{10}{\sqrt{0.14^2+10^2}}\frac{\si{\giga\electronvolt}}{\si{\giga\electronvolt}}\simeq 0.99990
\end{align*}
\begin{align*}
  \beta_{k^\pm}=\frac{10}{\sqrt{0.5^2+10^2}}\simeq0.99875
\end{align*}
なので
\begin{align*}
  0.99875<\frac{1}{n}\leq 0.99990\\
\end{align*}
\begin{align*}
  \therefore 1.00009\leq n <1.00125
\end{align*}
となる.ここで空気の屈折率は
\begin{align*}
  n=1+0.000273\frac{p}{\rm atm}
\end{align*}
で与えられるため
\begin{align*}
  0.358957\leq p\ /\ \si{atm}<4.5787
\end{align*}
の圧力範囲の空気を充填することで$\pi^\pm$と$K^\pm$を識別できる.
\mfig[width=8cm]{fig/fig1.jpg}{$1/n$の範囲}
\subsection*{(2)}
$1\ \si{GeV}$における$\pi^\pm$と$K^\pm$の$\beta$は
\begin{align*}
  \beta_{\pi^\pm}=\frac{1}{\sqrt{0.14^2+1^2}}\simeq0.990341
\end{align*}
\begin{align*}
  \beta_{k^\pm}=\frac{1}{\sqrt{0.5^2+1^2}}\simeq0.894427
\end{align*}
なのでSi単位系での速度は
\begin{align*}
  v_{\pi^\pm}=\beta_{\pi^\pm} c\simeq2.99896\times10^8\ \si{\metre.\second^{-1}}
\end{align*}
\begin{align*}
  v_{k^\pm}=\simeq2.68142\times10^8\ \si{\metre.\second^{-1}}
\end{align*}
したがって$1\ \si{\metre}$の検出器を通過するのに要する時間$t_{\pi^\pm}$, $t_{k^\pm}$はそれぞれ
\begin{align*}
  \Delta t_{\pi^\pm}=\frac{1}{v_{\pi^\pm}}\simeq3.36817\ \si{\nano\second}
\end{align*}
\begin{align*}
  \Delta t_{k^\pm}\simeq3.72936\ \si{\nano\second}
\end{align*}
であり
\begin{align*}
  |t_{\pi^\pm}-t_{k\pm}|\simeq361.188\ \si{\pico\second}
\end{align*}
となる.検出器の時間測定精度が$\sigma_t=50\ \si{\pico\second}$であることから,
始点と終点での2回の測定で最悪$100\ \si{\pico\second}$の誤差が入ったとしても
$1\ \si{GeV}$の$\pi^\pm$と$K^\pm$を識別できる.

次に$10\ \si{GeV}$の場合,前問の結果を用いれば
\begin{align*}
  v_{\pi^\pm}\simeq2.99763\times10^8\ \si{\metre.\second^{-1}}
\end{align*}
\begin{align*}
  v_{k^\pm}\simeq2.99418\times10^8\ \si{\metre.\second^{-1}}
\end{align*}
より
\begin{align*}
  \Delta t_{\pi^\pm}\simeq3.33596\ \si{\nano\second}
\end{align*}
\begin{align*}
  \Delta t_{k^\pm}\simeq3.33980\ \si{\nano\second}
\end{align*}
したがって
\begin{align*}
  |t_{\pi^\pm}-t_{k\pm}|\simeq3.84007\ \si{\pico\second}
\end{align*}
となる.検出器の時間測定精度が$\sigma_t=50\ \si{\pico\second}$であることから,この検出器では
$10\ \si{GeV}$の$\pi^\pm$と$K^\pm$を識別できない.
\bibliography{ref.bib}
\end{document}