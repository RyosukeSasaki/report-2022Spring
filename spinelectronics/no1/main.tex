\documentclass[uplatex,a4j,11pt,dvipdfmx]{jsarticle}
\usepackage{listings,jvlisting}
\bibliographystyle{junsrt}

\usepackage{url}

\usepackage{graphicx}
\usepackage{gnuplot-lua-tikz}
\usepackage{pgfplots}
\usepackage{tikz}
\usepackage{amsmath,amsfonts,amssymb}
\usepackage{bm}
\usepackage{siunitx}

\makeatletter
\def\fgcaption{\def\@captype{figure}\caption}
\makeatother
\newcommand{\setsections}[3]{
\setcounter{section}{#1}
\setcounter{subsection}{#2}
\setcounter{subsubsection}{#3}
}
\newcommand{\mfig}[3][width=15cm]{
\begin{center}
\includegraphics[#1]{#2}
\fgcaption{#3 \label{fig:#2}}
\end{center}
}
\newcommand{\gnu}[2]{
\begin{figure}[hptb]
\begin{center}
\input{#2}
\caption{#1}
\label{fig:#2}
\end{center}
\end{figure}
}

\begin{document}
\title{スピンエレクトロニクス No.1}
\author{61908697 佐々木良輔}
\date{}
\maketitle
静電エネルギーと静止質量エネルギーが等しいことから
\begin{align}
  m_ec^2&=\frac{1}{2}\frac{e^2}{4\pi\varepsilon_0}\frac{1}{a}\nonumber\\
  \therefore\ a&=\frac{e^2}{8\pi\varepsilon_0m_ec^2}
\end{align}
また電子スピンの角運動量は
\begin{align}
  \frac{2}{5}m_ea^2\omega&=\frac{\hbar}{2}\nonumber\\
  \therefore\ v=a\omega&=\frac{5\hbar}{4m_ea}
\end{align}
(2)に(1)を代入すると
\begin{align*}
  v&=\frac{5\hbar}{4m_e}\frac{8\pi\varepsilon_0m_ec^2}{e^2}\\
  &=\frac{10\hbar\pi\varepsilon_0c}{e^2}\times c\simeq 341c
\end{align*}
となり,スピン角運動量が電子の自転に由来しているとすると,その表面速度は光速を超えることがわかる.
\end{document}