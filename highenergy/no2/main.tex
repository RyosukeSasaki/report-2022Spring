\documentclass[uplatex,a4j,11pt,dvipdfmx]{jsarticle}
\usepackage{listings,jvlisting}
\bibliographystyle{junsrt}

\usepackage{url}

\usepackage{graphicx}
\usepackage{gnuplot-lua-tikz}
\usepackage{pgfplots}
\usepackage{tikz}
\usepackage{amsmath,amsfonts,amssymb}
\usepackage{bm}
\usepackage{siunitx}

\makeatletter
\def\fgcaption{\def\@captype{figure}\caption}
\makeatother
\newcommand{\setsections}[3]{
\setcounter{section}{#1}
\setcounter{subsection}{#2}
\setcounter{subsubsection}{#3}
}
\newcommand{\mfig}[3][width=15cm]{
\begin{center}
\includegraphics[#1]{#2}
\fgcaption{#3 \label{fig:#2}}
\end{center}
}
\newcommand{\gnu}[2]{
\begin{figure}[hptb]
\begin{center}
\input{#2}
\caption{#1}
\label{fig:#2}
\end{center}
\end{figure}
}

\begin{document}
\title{}
\author{61908697 佐々木良輔}
\date{}
\maketitle
\subsection*{問1}
系統誤差が独立であるとしてPredctionの誤差を合成すると
\begin{align}
  \sqrt{0.010^2+0.015^2}\simeq0.018
\end{align}
同様にDataの誤差については
\begin{align}
  \sqrt{0.007^2+0.020^2}\simeq0.021
\end{align}
となる.
\subsection*{問2}
\begin{align}
  f=\frac{n_{\rm up}}{n_{\rm down}}
\end{align}
とすると
\begin{align}
  \begin{split}
    \sigma_f&=\sqrt{\left(\frac{\partial f}{\partial n_{\rm up}}\right)^2\sigma_{\rm up}^2+
    \left(\frac{\partial f}{\partial n_{\rm down}}\right)^2\sigma_{\rm down}^2}\\
    &=\sqrt{\left(\frac{\sigma_{\rm up}}{n_{\rm down}}\right)^2+\left(\frac{n_{\rm up}}{n_{\rm down}^2}\sigma_{\rm down}\right)^2}\\
    &=\frac{n_{\rm up}}{n_{\rm down}}\sqrt{\left(\frac{\sigma_{\rm up}}{n_{\rm up}}\right)^2+\left(\frac{\sigma_{\rm down}}{n_{\rm down}}\right)^2}
  \end{split}
\end{align}
である.ここで$n_{\rm up}$, $n_{\rm down}$がそれぞれポアソン分布に従うとすると,
ポアソン分布の標準偏差は
\begin{align}
  \sigma=\sqrt{\mu}
\end{align}
なので
\begin{align}
  \sigma_{\rm up}=\sqrt{n_{\rm up}}\ \sigma_{\rm down}=\sqrt{n_{\rm down}}
\end{align}
したがって
\begin{align}
  \sigma_f&=\frac{n_{\rm up}}{n_{\rm down}}\sqrt{\frac{n_{\rm up}}{n_{\rm up}^2}+\frac{n_{\rm down}}{n_{\rm down}^2}}\\
  &=0.54\sqrt{\frac{139}{139^2}+\frac{256}{256^2}}=0.0568
\end{align}
となる.
\subsection*{問3}
観測結果が正規分布に従うとする.
ニュートリノ振動が無いときのUpとDownの比を1とすると観測結果は
\begin{align}
  \frac{1-\frac{139}{256}}{0.0568}=8.04
\end{align}
したがって$8.04\sigma$の有意性がある.
\subsection*{問4}
図では斜線の四角にMC+MC statとあり,MC法などを用いた予測値であると推測する.
また実験データは天頂角の余弦を5分割しており,角セグメントでのデータを用いて尤度を計算するとする.
すると全体の正規化に1自由度を要するとして,残り4自由度で推定を行ったと考えられる.
\bibliography{ref.bib}
\end{document}