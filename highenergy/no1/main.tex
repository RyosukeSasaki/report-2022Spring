\documentclass[leqno,uplatex,a4j,11pt,dvipdfmx]{jsarticle}
\usepackage{listings,jvlisting}
\bibliographystyle{junsrt}

\usepackage{url}

\usepackage{graphicx}
\usepackage{gnuplot-lua-tikz}
\usepackage{pgfplots}
\usepackage{tikz}
\usepackage{amsmath,amsfonts,amssymb}
\usepackage{bm}
\usepackage{siunitx}

\makeatletter
\def\fgcaption{\def\@captype{figure}\caption}
\makeatother
\newcommand{\setsections}[3]{
\setcounter{section}{#1}
\setcounter{subsection}{#2}
\setcounter{subsubsection}{#3}
}
\newcommand{\mfig}[3][width=15cm]{
\begin{center}
\includegraphics[#1]{#2}
\fgcaption{#3 \label{fig:#2}}
\end{center}
}
\newcommand{\gnu}[2]{
\begin{figure}[hptb]
\begin{center}
\input{#2}
\caption{#1}
\label{fig:#2}
\end{center}
\end{figure}
}

\begin{document}
\title{高エネルギー物理学 課題1}
\author{61908697 佐々木良輔}
\date{}
\maketitle
Bhabha散乱のFeynman図は図\ref{fig:fig/feynman.png}の2種類である.
(a)について,運動量保存$q=p_1-p_3=p_4-p_2$をもちいて
\begin{align*}
  (2\pi)^4\delta(p_1-p_3-q)=(2\pi)^4\delta(q+p_2-p_4)=(2\pi)^4\delta(p_1-p_3+p_2-p_4)
\end{align*}
と表される.したがって対応する式は
\begin{align*}
  \int\frac{d^4q}{(2\pi)^4}\overline{u}(p_3)ie\gamma^\mu u(p_1)(2\pi)^4&\delta(p_1-p_3-q)\frac{-i\eta_{\mu\nu}}{q^2}\overline{v}(p_2)ie\gamma^\nu v(p_4)\\
  =&\ \overline{u}(p_3)ie\gamma^\mu u(p_1)\frac{-i\eta_{\mu\nu}}{(p_1-p_3)^2}\overline{v}(p_2)ie\gamma^\nu v(p_4)
\end{align*}
となる.次に(b)について,運動量保存$q=p_1-p_2=p_4-p_3$をもちいて
\begin{align*}
  (2\pi)^4\delta(p_1-p_2-q)=(2\pi)^4\delta(q+p_3-p_4)=(2\pi)^4\delta(p_1-p_2+p_3-p_4)
\end{align*}
と表される.したがって対応する式は
\begin{align*}
  -\int\frac{d^4q}{(2\pi)^4}\overline{v}(p_2)ie\gamma^\mu u(p_1)(2\pi)^4&\delta(p_1-p_2-q)\frac{-i\eta_{\mu\nu}}{q^2}\overline{u}(p_3)ie\gamma^\nu v(p_4)\\
  =&\ \overline{v}(p_2)ie\gamma^\mu u(p_1)\frac{i\eta_{\mu\nu}}{(p_1-p_2)^2}\overline{u}(p_3)ie\gamma^\nu v(p_4)\\
\end{align*}
ここで上の式は(a)の場合に対して$p_2\leftrightarrow p_3$と交換しているので,
fermionの反交換関係から$-$符号をつけた.
したがって不変散乱振幅は
\begin{align*}
  i\mathcal{M}=&\overline{u}(p_3)ie\gamma^\mu u(p_1)\frac{-i\eta_{\mu\nu}}{(p_1-p_3)^2}\overline{v}(p_2)ie\gamma^\nu v(p_4)\\
  &\ +\overline{v}(p_2)ie\gamma^\mu u(p_1)\frac{i\eta_{\mu\nu}}{(p_1-p_2)^2}\overline{u}(p_3)ie\gamma^\nu v(p_4)
\end{align*}
となる.
\mfig[width=12cm]{fig/feynman.png}{Bhabha散乱のFeynman図}
\bibliography{ref.bib}
\end{document}